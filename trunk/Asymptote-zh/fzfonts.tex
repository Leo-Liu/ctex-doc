\usepackage[BoldFont,SlantFont,CJKnumber]{xeCJK}
\usepackage{CJKfntef}

% 方正系列字体设置
\setCJKmainfont[BoldFont={FZXiaoBiaoSong-B05}, ItalicFont={FZKai-Z03}]{FZShuSong_GB18030-Z01} % 书宋、小标宋、楷体
\setCJKsansfont[BoldFont={FZHei-B01}]{FZXiHei I-Z08} % 细黑、黑体
\setCJKmonofont{FZFangSong-Z02} % 仿宋

\setCJKfamilyfont{rm}[BoldFont={FZXiaoBiaoSong-B05}, ItalicFont={FZKai-Z03}]{FZShuSong_GB18030-Z01} % 同 main
\setCJKfamilyfont{sf}[BoldFont={FZHei-B01}]{FZXiHei I-Z08} % 同 sans
\setCJKfamilyfont{tt}{FZFangSong-Z02} % 同 mono

\setCJKfamilyfont{song}[BoldFont={FZXiaoBiaoSong-B05}]{FZShuSong_GB18030-Z01} % 书宋和小标宋(除无楷体外同 main)
\setCJKfamilyfont{kai}{FZKai-Z03} % 楷体
\setCJKfamilyfont{hei}[BoldFont={FZHei-B01}]{FZXiHei I-Z08} % 细黑、黑体(同 sans)
\setCJKfamilyfont{fs}{FZFangSong-Z02} % 仿宋(同 mono)

\setCJKfamilyfont{you}[BoldFont={FZZhunYuan-M02}]{FZXiYuan-M01} % 细圆、准圆
\setCJKfamilyfont{li}[ItalicFont={FZLiShu II-S06}]{FZLiShu-S01} % 隶书、隶二(italic)

\def\heiti{\CJKfamily{hei}\bfseries}

