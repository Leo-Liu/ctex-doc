\chapter{简介}

\begin{quotation}
滚滚长江东逝水,浪花淘尽英雄。是非成败转头空。青山依旧在,几度夕阳红。

白发渔樵江渚上,惯看秋月春风。一壶浊酒喜相逢。古今多少事,都付笑谈中。
\begin{flushright}
---~杨慎《临江仙》
\end{flushright}
\end{quotation}

\section{历史回顾}

\LaTeX~是一种面向数学和其它科技文档的电子排版系统。一般人们提到的~\LaTeX~是一个总称,它包括~\TeX、\LaTeX、\AmS-\LaTeX~等\footnote{一般认为~\TeX~是一种引擎,\LaTeX~是一种格式,而~\AmS-\LaTeX~等是宏集。此处目的是简介,故不展开讨论。}。

\TeX~的开发始于~1977~年~5~月,Donald E. Knuth\footnote{斯坦福大学计算机系教授,已退休。}开发它的初衷是用于《The Art of Computer Programming》的排版。1962~年~Knuth~开始写一本关于编译器设计的书,原计划是~12~章的单行本。不久~Knuth~觉得此书涉及的领域应该扩大,于是越写越多,如滔滔江水连绵不绝,又如黄河泛滥一发不可收拾。1965~年完成的初稿居然有~3000~页,全是手写的!据出版商估计,这些手稿印刷出来需要~2000~页,出书的计划只好改为七卷,每卷一或两章。1976~年~Knuth~改写第二卷的第二版时,很郁闷地发现第一卷的铅版不见了,而当时电子排版刚刚兴起,质量还差强人意。于是~Knuth~仰天长啸:“我要扼住命运的咽喉”,决定自己开发一个全新的系统,这就是~\TeX。

1978~年~\TeX~第一版发布后好评如潮,Knuth~趁热打铁在~1982~年发布了第二版。人们现在使用的~\TeX~基本就是第二版,中间只有一些小的改进。1990~年~\TeX~ v3.0~发布后,Knuth~宣布除了修正~bug~外停止~\TeX~的开发,因为他要集中精力完成那本巨著的后几卷\footnote{已出版的前三卷是:《Fundamental Algorithms》、《Seminumerical Algorithms》、《Sorting and Searching》;第四卷《Combinatorial Algorithms》和第五卷《Syntactic Algorithms》正在写作中,预计~2015~年出版;第六卷《Theory of Context-free Languages》和第七卷《Compiler Techniques》尚未安排上工作日程。}。此后每发布一个修正版,版本号就增加一位小数,使得它趋近于$\pi$(目前是~3.141592)。Knuth~希望将来他离世时,\TeX~的版本号永远固定下来,从此人们不再改动他的代码。他开发的另一个软件~\MF~也作类似处理,它的版本号趋近于$e$,目前是~2.71828。

\TeX~是一种语言也是一个宏处理器,这使得它很好很强大,但是它同时又很繁琐,让人难以接近。因此~Knuth~提供了一个对~\TeX~进行了封装的宏集~Plain \TeX,里面有一些高级命令,有了它最终用户就无须直接面对枯燥无味的~\TeX。

然而~Plain \TeX~还是不够高级,所以~Leslie Lamport\footnote{现供职于微软研究院。}在~80~年代初期开发了另一个基于~\TeX~的宏集~\LaTeX。1992~年~\LaTeX~ v2.09~发布后,Lamport~退居二线,之后的开发活动由~Frank Mittelbach~领导的~The LaTeX Team~接管。此小组发布的最后版本是~1994~年的~\LaTeXe,他们同时还在进行~\LaTeX~3~的开发,只是正式版看起来遥遥无期。

起初,美国数学学会(American Mathematical Society,AMS)看着\TeX~是好的,就派~Michael Spivak~写了~\AmS-\TeX,这项基于~Plain \TeX~的开发活动进行了两年(1983--1985)。后来与时俱进的~AMS~又看着~\LaTeX~是好的,就想转移阵地,但是他们的字体遇到了麻烦。恰好~Mittelbach~和~Rainer Schöpf(后者也是~LaTeX Team~的成员)刚刚发布了~New Font Selection Scheme for \LaTeX(NFSS),AMS~看着还不错,就拜托他们把~AMSFonts加入~\LaTeX,继而在~1989~年请他们开发~\AmS-\LaTeX。\AmS-\LaTeX~发布于~1990~年,之后它被整合为~\AmS~宏包,像其它宏包一样可以直接运行于~\LaTeX。

\section{优点和缺点}
当前的文字处理系统大致可以分为两种:标记语言(Markup Language)式的,比如\LaTeX;所见即所得(WYSIWYG)式的,比如~MS Word\footnote{其实~Word~也有自己的标记语言域代码(field code),只是一般用户不了解。}。

一般而言,\LaTeX~相对于所见即所得系统有如下优点:
\begin{itemize}
    \item 高质量\ 它制作的版面看起来更专业,数学公式尤其赏心悦目。
    \item 结构化\ 它的文档结构清晰。
    \item 批处理\ 它的源文件是文本文件,便于批处理,虽然解释(parse)源文件可能很费劲。
    \item 跨平台\ 它几乎可以运行于所有电脑硬件和操作系统平台。
    \item 免费\ 多数~\LaTeX~软件都是免费的,虽然也有一些商业软件。
\end{itemize}

相应地,\LaTeX~的工作流程、设计原则,资源的缺乏,以及开发人员的历史局限性等种种原因也导致了一些缺陷:
\begin{itemize}
    \item 制作过程繁琐,有时需要反复编译,不能直接或实时看到结果。
    \item 宏包鱼龙混杂,水准参差不齐,风格不够统一。
    \item 排版风格比较统一,但因而缺乏灵活性。
    \item 用户支持不够好,文档不完善。
    \item 对国际语言和字体的支持很差。
\end{itemize}

抛开~MS Word~不谈,即使跟同为标记语言的~HTML/Web~系统相比,\LaTeX~也有一些不足之处。比如~Web~浏览器对~HTML~内容的渲染(render)比~DVI~浏览器对~\LaTeX~内容的渲染要快上许多,基本上可以算是实时。虽然~HTML~内容可能没有~LaTeX~那么复杂,但是~DVI~毕竟是已经被~\LaTeX~编译过的格式。

还有一点令人困惑的是,有一部分~\LaTeX~阵营的人士习惯于称对方为“邪恶的”或“出卖灵魂的”,如果昂贵的微软系统应当为人诟病,那么更贵的苹果系统为何却被人追捧?

2000~年有记者在采访~Lamport~时问:“为什么当前没有高质量的所见即所得排版系统?”他回答道:“门槛太高了,一个所见即所得系统要做到~\LaTeX~当前的水平,工作量之大不是单枪匹马所能完成\footnote{\TeX/\LaTeX~也不单单是那几个大腕儿完成的,他们背后还有众多默默无闻的小人物,比如当年~Knuth~手下的大批学生。此所谓一将功成万骨枯。}。微软那样的大公司可以做,但是市场太小了。我偶尔也会想加入“Dark Side”,让微软给我一组人马来开发一个这样的系统。”(包老师注:他果然于次年加入微软。)

窃以为这两大阵营其实是萝卜青菜的关系,与其抱残守缺、互相攻讦,不如各取所需;甚至可以捐弃前嫌、取长补短,共建和谐社会。

\section{软件准备}
\label{sec:latexsoft}

\LaTeX~是一个软件系统,同时也是一套标准。遵照这些标准,实现了(implement)所要求功能的软件集合被称为发行版(distribution)。与此类似的例子有~Java~和~Linux,比如SUN、IBM、BEA~等公司都有自己的~Java~虚拟机(JVM),它们都被称作~Java~的实现;而Linux~有~Red Hat/~Fedora、Ubuntu、SuSE~等众多的发行版。

\begin{table}[htbp]
\caption{\LaTeX~发行版与编辑器}
\label{tab:latexsoft}
\centering
\begin{tabular}{lll}
    \toprule
    操作系统 & 发行版 & 编辑器 \\
    \midrule
    Windows & \href{http://www.miktex.org/}{MikTeX} & \href{http://www.toolscenter.org/}{TeXnicCenter}、\href{http://www.winedt.com/}{WinEdt} \\
    Unix/Linux & \href{http://www.tug.org/texlive/}{TeX Live} & \href{http://www.gnu.org/software/emacs/emacs.html}{Emacs}、\href{http://vim.sourceforge.net/}{vim}、\href{http://kile.sourceforge.net/}{Kile} \\
    Mac OS & \href{http://www.tug.org/mactex/}{MacTeX} & \href{http://www.uoregon.edu/~koch/texshop/}{TeXShop} \\
    \bottomrule
\end{tabular}
\end{table}

\LaTeX~发行版只提供了一个~\LaTeX~后台处理机制,用户还需要一个前台编辑器来编辑它的源文件。常用的~\LaTeX~发行版和编辑器见\fref{tab:latexsoft}。在使用~\LaTeX~的过程中可能还需要其它一些软件,将在后面相关章节中分别介绍。

\section{学习方法}
\begin{quotation}
在科学上没有平坦的大道,只有那些不畏劳苦沿着陡峭山路攀登的人,才有希望达到光辉的顶点。
\begin{flushright}
---~卡尔·马克思
\end{flushright}
\end{quotation}

\begin{quotation}
无他,唯手熟尔。
\begin{flushright}
---~卖油翁
\end{flushright}
\end{quotation}

\begin{quotation}
用心。
\begin{flushright}
---~斯蒂芬·周
\end{flushright}
\end{quotation}

限于篇幅和水平,本文只能提供一个概览外加一些八卦。比较严谨的入门资料有~Tobias Oetiker~的《A (Not So) Short Introduction to \LaTeXe》\citep{Oetiker_2008}(简称lshort);若想对~\LaTeX~有更深入全面的了解,可以拜读~Mittelbach~的《The \LaTeX~ Companion》\citep{Mittelbach_2004}。

中文资料可参考李果正的《大家来学~\LaTeX》\citep{Lee_2004},lshort~有吴凌云等人翻译的中文版本\footnote{此译本首发于~CTeX~论坛,但是需要注册才能看见链接,所以请读者自行搜索。}。

\href{http://www.ctan.org/}{Comprehensive TeX Archive Network}(CTAN)和~\href{http://www.tug.org/}{TeX Users Group}~(TUG)提供了权威、丰富的资源。

\href{http://www.text.ac.uk/}{英国TUG}~和~\href{http://www.ctex.org}{CTeX}~分别提供了常见问题集(FAQ)\citep{UKTUG_FAQ,CTeX_FAQ},一般问题多会在这里找到答案。

中文~\TeX~论坛有\href{http://www.smth.org/bbsdoc.php?board=TeX}{水木清华~BBS TeX~版}、\href{http://bbs.ctex.org/}{CTeX~论坛}。

\bibliographystyle{unsrtnat}
\bibliography{reading}
\newpage
