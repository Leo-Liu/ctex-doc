\chapter{序}

\begin{quotation}
满纸荒唐言,一把辛酸泪!都云作者痴,谁解其中味?\footnote{当年作博士论文时虽不曾增删五次披阅十载,也被折磨得欲仙欲死,故与室友戏言将此五绝加入序言。多年以后的今天终于实现了此夙愿。}
\begin{flushright}
--- 曹雪芹
\end{flushright}
\end{quotation}

最早听说~\LaTeX~大约是~2002~年,一位同事演示了用它排版的一篇文章和几幅图。包老师\footnote{吾有多重人格,比如本色的是阿黄,下围棋的是隐忍灰衣人,道貌岸然的是包老师。}不以为然,因为那些东西用~Microsoft Word~和~Visio~也可以做到,而且可以做得更快。再次听说它是王垠同学在闹退学,传说他玩~Linux~和~\LaTeX~而走火入魔。

大约是~2005~年底,看了一下~lshort,用~\LaTeX~记了些数学笔记,开始有点感觉。包老师生性愚钝,所以喜欢相对简单的东西。HTML、Java~都用手写,FrontPage、Dreamweaver、JBuilder~之类笨重的家伙看两眼就扔了,所以喜欢上~\LaTeX~只是时间问题。

次年老妻要写博士论文,拿出~Word~底稿让我排版。大家都知道~Word~太简单了,谁都能用,但是不是谁都能用好。人称电脑杀手的老妻制作的~Word~文档自然使出了各种奇门遁甲,加上她实验室、学校和家里电脑里的三个~EndNote~版本互不兼容,实在难以驯服。我只好重起炉灶,拿她的博士论文当小白鼠,试验一下~\LaTeX~的威力。

就这样接触了两三年,总算略窥门径,感觉~\LaTeX~实在是博大精深,浩如烟海。而人到中年大脑储存空间和处理能力都有点捉襟见肘,故时常作些笔记。一来对常用资料和问题进行汇编索引,便于查询;二来也记录一些心得。

日前老妻吵着要学~\LaTeX,便想这份笔记对初学者或有些许借鉴意义,于是系统地整理了一番,添油加醋包装上市。

原本打算分九章,以纪念《九章算术》,实际上第八章完成时已如强弩之末,最后一章还须另择黄道吉日。

本文第一章谈谈历史背景;第二章介绍入门基础;第三至五章讲解数学、插图、表格等对象的用法;第六章是一些特殊功能;第七、八章讨论中文和字体的处理;第九章附加定制内容。

从难易程度上看前两章较简单,插图、字体两章较难。一般认为~\LaTeX~相对于微软的傻瓜型软件比较难学,所以这里采取循序渐进,温水煮青蛙的方法。

初则示弱,麻痹读者;再则巧言令色,请君入瓮;三则舌绽莲花,诱敌深入;彼入得罄中则摧动机关,关门打狗;继而严刑拷打,痛加折磨;待其意乱情迷彷徨无计之时,给予当头棒喝醍醐灌顶,虽戛然而止亦余音绕梁。

鄙人才疏学浅功力不逮,面对汗牛充栋罄竹难书\footnote{此处用法循阿扁古例。}的资料,未免考虑不周挂一漏万,或有误导,敬请海涵。若有高手高手高高手略拨闲暇指点一二,在下感激不尽\footnote{\href{mailto:huang.xingang@gmail.com}{huang.xingang@gmail.com}}
。

借此感谢一下老妻,如果不是伊天天看韩剧,包老师也不会有时间灌水和整理这份笔记。
