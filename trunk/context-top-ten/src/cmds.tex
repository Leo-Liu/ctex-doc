%% Time-stamp: <corsair 2008-11-12 14:01:28>
%% 
%% Author: Corsair <chris.corsair@gmail.com>
%% Released under GNU FDL v1.3 (http://www.gnu.org/copyleft/fdl.html)

\usemodule[pre-general]
\setuppapersize[S6][S6]
\setuplayout [width=fit,
              rightmargin=1.5cm,
              leftmargin=1.5cm,
              leftmargindistance=0pt,
              rightmargindistance=0pt,
              height=fit,
              topspace=1.8cm,
              header=0pt,
              footer=1.2cm,
              bottomspace=0.5cm,
              backspace=1.5cm,
              location=singlesided]

% \starttypescript[serif][palatino][palatino]
%   \definefontsynonym[Serif][pplr]
% \stoptypescript

\definecolor[orange][r=0.8047, g=0.3594, b=0]
\definecolor[problue][r=0.125, g=0.28906, b=0.52734]
\definecolor[litegrey][r=0.9, g=0.9, b=0.9]

% XeTeX Fonts
% Linux
\definetypeface[prof][ss][Xsans][Myriad Pro]
\definetypeface[prof][rm][Xserif][Palatino LT Std]
\definetypeface[prof][tt][Xmono][Consolas]
% Mac
% \definetypeface[prof][ss][Xsans][Myriad Pro]
% \definetypeface[prof][rm][Xserif][Palatino]
% \definetypeface[prof][tt][Xmono][Monaco][default][rscale=0.85]
\setupbodyfont
  [prof,14.4pt]
\setuphead
  [section]
  [numberstyle=\ss\bf,
    textstyle=\ss\bf,
    page=yes]
\setuphead
  [subject]
  [textstyle=\ss\bf,color=sub]
\setupcaptions[style=\ss]
\setuptyping
  [before=\startbackground,
    after=\stopbackground]
\setupbackground[background=color, backgroundcolor=litegrey]
\setuptype
  [style=\tt, color=problue]
\setupitemize[packed]

% \setuppagenumbering[state=stop]
% \switchtobodyfont[modern,20pt]
\setuptolerance[verytolerant,stretch]
\setupinteractionscreen[option=max]
\setupwhitespace[medium]
\setupindenting[medium]
\setupinterlinespace[line=1.3\bodyfontsize]
\setupfootertexts[pagenumber]

\setupcolors[state=start]
\setupbackgrounds[page][background=color,backgroundcolor=]
\setupinteraction[state=start,color=middlered]
\useURL [garden][http://contextgarden.net][][Context garden]
\useURL [wiki][http://wiki.contextgarden.net/Main_Page][][\CONTEXT\
  wiki]
\useURL [wiki_table][http://wiki.contextgarden.net/Table][][table page]
\definetyping[TEX][option=TEX,style=type]

\defineoverlay
  [bg]
  [{\externalfigure[bg-slide.pdf]}]

\defineoverlay
  [title]
  [{\externalfigure[title-slide.pdf][width=\overlaywidth,height=\overlayheight]}]

\definedescription[definition][location=left]

\starttext
\setupbackgrounds[paper][state=start,background=bg]
% \setupfooter[state=start]

% Title Slide
\page[blank]
\setupbackgrounds[paper][state=start,background=title]

\ \par
\blank[3cm]
\switchtobodyfont[sansserif,1.2cm]
\midaligned{\bf \CONTEXT\ Macros Top Ten}
\switchtobodyfont[sansserif,0.8cm]
% \midaligned{\bf An Introduction to \CONTEXT\ (draft)}
\blank[2cm]
\midaligned{Corsair {\tt\tx <chris.corsair@gmail.com>}}
\page[yes]
\setupbackgrounds[paper][state=stop]
\setupbackgrounds[paper][state=start,background=bg]
\switchtobodyfont[prof,14.4pt]

%% End of Title
\rm

\completecontent % [alternative=f]

% \setuppagenumbering[state=start]

\section{Introduction}

\CONTEXT\ has a huge quantity of commands which might shoo away
novices.  For example, there are at least 9 commands to change
characters' case and typeset them properly, including \type{\cap},
\type{\Cap}, \type{\kap}, \type{\Words}, etc..  So in the beginning of
the \CONTEXT\ manual, the author lists ten groups of them which are
considered most frequently used:

  \startitemize[n]
  \item \type{\starttext}, \type{stoptext}
  \item \type{\chapter}, \type{\section}, \type{\title},
    \type{\subject}, \type{\setuphead}, \type{\completecontent}
  \item \type{\em}, \type{\bf}, \type{\cap}
  \item \type{\startitemize}, \type{\stopitemize}, 
    \type{\item}, \type{\head}
  \item \type{\abbreviation}, \type{\infull}, 
    \type{\completelistofabbreviations}
  \item \type{\placefigure}, \type{\externalfigure}, 
    \type{\useexternalfigures}
  \item \type{\placetable}, \type{\starttable}, \type{\stoptable}
  \item \type{\definedescription}, \type{\defineenumeration}
  \item \type{\index}, \type{\completeindex}
  \item \type{\setuplayout}, \type{\setupfootertexts}, \type{\setupheadertexts}
  \stopitemize

So let's walk 'em through!

\section{The \type{text} Environment}

You just cannot typeset anything if not use this environment.  It's
like the \type{document} environment in \LaTeX.  In \CONTEXT, this
(and all) environment is issued with \type{\start} and \type{\stop}.

\startTEX
\starttext
Once there was a mountain ...  
\stoptext
\stopTEX

Like in \LaTeX, it is preferred to setup the whole document before
\type{\starttext} so that structure and style are separated.

\section[sec:struct]{Structure of An Article}

You {\em do} know that articles have structure, don't you?  Remember
when we were young and the world was in chaos, we used to typeset a
line of title in M\$ Word, by bold-ing, large-ing and centering it?
Yes, but that was bad old time.  Today we use \type{\section} and
\type{\title}!

\startTEX
\starttext
  \section[sec:mnt]{The Mountain}
  Once there was a mountain ...
\stoptext
\stopTEX

\startframedtext[middle]
  {\tfa \bf 1 \ \ The Mountain}
  \blank[0.5cm]
  Once there was a mountain \ldots
\stopframedtext

The \type{[...]} stuff is an optional label of this head, which
enables you to cross-reference it easily with \type{\in}, \type{\at}
or \type{\about}.

\type{\subject} is another unnumbered version of \type{\section}.
Some of the structure elements are listed in table~\in[tab:strele].

\placetable
  [here][tab:strele]
  {\Words{Structure elements in \CONTEXT}}
  \starttable[|c|l|l|]
    \NC level \VL numbered title \VL unnumbered title \LR
    \HL
    \NC 1 \VL \type{\part} \VL \AR
    \NC 2 \VL \type{\chapter} \VL \type{\title} \AR
    \NC 3 \VL \type{\section} \VL \type{\subject} \AR
    \NC 4 \VL \type{\subsection} \VL \type{\subsubject} \AR
    \NC 5 \VL \type{\subsubsection} \VL \type{\subsubsubject} \AR
  \stoptable

  Now we know how to typeset heads.  But what if we want to make all
  heads in a different font, for example, change them to sans-serif
  font (like in this artile)?  Must we add a \type{\ss} in every
  section, subsection, subsubsection ?  Definitely no.  \CONTEXT\
  provides a rather convenient interface to change whatever style,
  including that of titles, in your article.

\startTEX
\setuphead
  [section,subsection,subsubsection]
  [numberstyle=\ss\bf, textstyle=\ss\bf]
\stopTEX

Put this block of code before \type{\starttext}, and boom, we're
done.  In fact, in \CONTEXT, almost everything can be setup in this
way.  \pagereference[page:itemize] Take \type{itemize} environment as example which is used to
typeset a list, \type{\setupitemize[packed]} makes all lists be
rendered in a compact style.

Structuring elements have a bunch of other options, which are
explained in {\it \CONTEXT en} (\CONTEXT\ manual), section 8.2.

With these structures, we can form a table of contents for out article
with \type{\completecontent} which is similar to
\type{\tableofcontents} in \LaTeX, except that the former one is much
much more customizable.

\section[sec:font]{Fancy Fonts}

Why do we use different fonts in the main text?  Well, we have our
reasons.  The most common one is to emphasize something.  In \CONTEXT\
(actually in plain \TeX) we use \type{\em} to do it.  It does various
things in different environment.  Normally, it
toggles “italicity” of the text (or make slanted when the text is
already italic?), while in quoted text, it might underline it.

\startTEX
Once there was a {\em mountain}, 
in the mountain there {\it was a {\em temple}},
in which a {\bf {\em monk} was telling a story}.
\stopTEX

\startframedtext[middle]
Once there was a {\em mountain}, 
in the {\it mountain {\em there} was a temple},
in which a {\bf {\em monk} was telling a story}.
\stopframedtext

So why bother using \type{\em} instead of \type{\it}?  Simple.
\type{\em} tells us the text there is important while \type{\it} does
not.  If we use \type{\it} in all cases, what so important about a
important block of text comparing to a book's name?

Besides italic and bold fonts, \TeX\ make heavy use of \cap{small cap}
characters which are look like small upper-cased characters, though in
theory they are {\em not} upper-cased characters in small size.  \TeX
ers love small cap font because it looks really nice in some font
families that come with their \TeX\ distributions.  In fact, small cap
face in those families are separately designed just like italic and
bold faces.

\pagereference[page:sc]
Generally, small cap is used to indicate that a word is an
abbreviation or an acronym.  For example, “\cap{pdf}” is produced from
\type|\cap{pdf}|, and “\cap{html}” from \type|\cap{html}|.

\section[sec:lists]{Lists}

Lists help us thinking.  There are at least two kinds of list existing
in this universe: numbered and unnumbered.  In \LaTeX, we use
\type{itemize} and \type{enumerate} environments respectively, while
in \CONTEXT, \type{itemize} rules them all.

\startTEX
\startitemize[n]
\item Atom Heart Mother
\item Dark Side of The Moon
\item Wish You Were Here
\stopitemize
\stopTEX

\startframedtext[middle]
  \startitemize[n,unpacked]
  \item Atom Heart Mother
  \item Dark Side of The Moon
  \item Wish You Were Here
  \stopitemize
\stopframedtext

The \type{[n]} tells \CONTEXT\ that this list is a numbered one.
Other options are listed in table \in[tab:listprefix].

\placetable
  [here]
  [tab:listprefix]
  {\Words{Table of all list prefixes}}
  \starttable[|c|l|c|l|]
    \NC\ \bf option \NC \bf prefix \VL \bf option \NC \bf prefix \LR
    \HL
    \NC \tt n \NC 1,2,3 \VL \tt 1 \NC dot \AR
    \NC \tt a \NC a,b,c \VL \tt 2 \NC dash \AR
    \NC \tt A \NC A,B,C \VL \tt 3 \NC star \AR
    \NC \tt KA \NC \kap{a,b,c} \VL \tt 4 \NC triangle \AR
    \NC \tt r \NC \romannumeral1,\romannumeral2,\romannumeral3 \VL \tt
    5 \NC circle \AR
    \NC \tt R \NC
    \uppercase\expandafter{\romannumeral1},\uppercase\expandafter{\romannumeral2},\uppercase\expandafter{\romannumeral3}
        \VL \tt 6 \NC big circle \AR
    \NC \tt KR \NC \cap{\romannumeral1,\romannumeral2,\romannumeral3} \VL
        \tt 7 \NC bigger circle \AR
    \NC \tt m \NC {\os 1},{\os 2},{\os 3} \VL \tt 8 \NC square \AR
    \NC \tt g \NC $\alpha$,$\beta$,$\gamma$ \VL \NC \AR
    \NC \tt G \NC $\Alpha$,$\Beta$,$\Gamma$ \VL \NC \AR
  \stoptable

Yes, every possible prefix you can think of is here for you.  And if
you think that space between lines is way too large, add a
\type{packed} option to \type{\startitemize} or set it up in the way I
discribed in section \in[sec:struct], page \at[page:itemize].

There's a \type{\head} macro also, to typeset fancy two-leveled lists
like the following:

\startTEX
\startitemize[1,packed]
  \head Pink Floyd \par
  \startitemize[g]
  \item Atom Heart Mother
  \item Dark Side of The Moon
  \item Wish You Were Here
  \stopitemize
  \head Muse \par
  \startitemize[continue]
  \item Black Holes and Revelations
  \item Absolution
  \item Hullabaloo
  \stopitemize
\stopitemize
\stopTEX

\startframedtext[middle]
\startitemize[1,packed]
  \head Pink Floyd \par
  \startitemize[g]
  \item Atom Heart Mother
  \item Dark Side of The Moon
  \item Wish You Were Here
  \stopitemize
  \head Muse \par
  \startitemize[continue]
  \item Black Holes and Revelations
  \item Absolution
  \item Hullabaloo
  \stopitemize
\stopitemize
\stopframedtext

\section{Long Things Short}

\definesynonyms[abbr][abbrs][\infull]

In section \in[sec:font], page \at[page:sc], we referred that usually
people use small cap font to indicate a synonym or abbreviation.
However, if you are writing a really formal document like a science
paper, you might want to have more control over synonyms and even had
a index of them.  In \CONTEXT, we can do this with
\type{\definesynonyms} and \type{\completelistofabbreviations}.

First off, we define a new type of synonym called \type{abbr}:

\startTEX
\definesynonyms[abbr][abbrs][\infull]
\stopTEX

The three arguments are all arbitrary.  The first two are singular and
plural name of this kind of synonym, and the third one is the command
that we will use to expand the abbreviations.

Now we define an \type{abbr}:

\startTEX
\abbr{CSS}{Cascading Style Sheets}
\stopTEX

Thus we can use this abbreviation in our article:

\startTEX
\infull{CSS} is a way of rendering the style of a web page from multiple sources with a defined order of precedence where the definitions of any style element conflict.
\stopTEX

\startframedtext[middle]
  \abbr{CSS}{Cascading Style Sheets}

  \infull{CSS} is a way of rendering the style of a web page from
  multiple sources with a defined order of precedence where the
  definitions of any style element conflict.
\stopframedtext

The newly defined macro \type{\abbr} has another usage, which can
define a macro to make our fingers comfortable:

\startTEX
\abbr[css]{CSS}{Cascading Style Sheets}
\stopTEX

Thus, \type{\css} does the same things as \type|\infull{CSS}| does.

Also, \type{\definesynonyms} has another form:

\startTEX
\definesynonyms[abbr][abbrs][\inshort][\infull]
\stopTEX

Then we have a new macro \type{\inshort} which just leave the
abbreviation be.

\startTEX
\infull{CSS} (\inshort{CSS}) is a way of rendering the style of a web page from multiple sources with a defined order of precedence where the definitions of any style element conflict.
\stopTEX

\definesynonyms[abbr][abbrs][\inshort][\infull]
\abbr{CSS}{Cascading Style Sheets}

\startframedtext[middle]
  \infull{CSS} (\inshort{CSS}) is a way of rendering the style of a
  web page from multiple sources with a defined order of precedence
  where the definitions of any style element conflict.
\stopframedtext

Like the table of contents, we can have a list of all \type{abbr}s:

\startTEX
\completelistofabbrs
\stopTEX

In case you don't know, “abbrs” in the macro above is the plural
name we defined with \type{\definesynonyms}.

\section{Figures Talk}

If you have used \LaTeX\ before, you might have noticed how hard to
work with figures in it.  It even needs an external macro package to
do that!  But in \CONTEXT, we have \type{\externalfigure}.

\startTEX
\externalfigure[sin.pdf][width=7cm]
\stopTEX

\externalfigure[sin.pdf][width=7cm]

That's it.  We inserted a figure into our article.  Normally \CONTEXT\
will identify same image file inserted multiple times, and reuse it to
save memory and reduce output file size.  If you are kind of in doubt
with it and want a fine control, \type{\useexternalfigure} is for you.

\startTEX
\useexternalfigure[sine][sin.pdf][width=3cm]
\externalfigure[sine]
\stopTEX

However, \TeX ers hardly like typesetting a figure in this way that
the picture is fixed there where the codes are.  More often do they
use a “floating environment” instead, in which position of a figure
is not fixed, but, in theory, “hanging around” in a specific area,
and the actual position is determined by an algorithm stored in the
\TeX\ compiler.  \CONTEXT\ provides \type{\placefigure} macro to place
a floating figure.

% So what's so great about this floating thing?  Simply put, have you
% ever run into such a situation that in M\$ Word, you put a figure just
% at the bottom of a page, and you made a horrible decision that you
% inserted a line of text before that figure, and then you found
% yourself in hell because the figure was pushed into the next page and
% its previous room became an area of blank which seemed to be laughing
% at you?  Do you know why we are not letting the figure be in a fixed
% position now?

\startTEX
\placefigure[here][fig:smile]{\Words{GNU post}}
  {\externalfigure[post.pdf][width=6cm]}
\stopTEX

\placefigure[here][fig:smile]{\Words{GNU post}}
  {\externalfigure[post.pdf][width=6cm]}

The \type{[here]} option tells \CONTEXT\ that we prefer this figure
right here if it is really fit.  It can also be \type{left},
\type{right}, \type{top}, \type{bottom}, \ldots  We can as well ignore
this option which leaves \type{here} as default.

Furthermore, multiple images can be placed in one block using the
\type{combination} environment.

\startTEX
\placefigure
  [bottom]
  [fig:comb]
  {\Words{Multiple figures in one floating env.}}
  {\startcombination[3*2]
     {\externalfigure[post.pdf][width=2.5cm]} {a} 
     {\externalfigure[fish.pdf][width=2.5cm]} {b}
     {\externalfigure[apple.pdf][width=2cm]} {c} 
     {\externalfigure[gnu.pdf][width=2.5cm]} {d}
     {\externalfigure[hacker.pdf][width=2.5cm]} {e}
     {\externalfigure[ctanlion.pdf][width=2.8cm]} {f}
   \stopcombination}
\stopTEX

\placefigure
  [bottom]
  [fig:comb]
  {\Words{Multiple figures in one floating env.}}
  {\startcombination[3*2]
     {\externalfigure[post.pdf][width=2.5cm]} {a} 
     {\externalfigure[fish.pdf][width=2.5cm]} {b}
     {\externalfigure[apple.pdf][width=2cm]} {c} 
     {\externalfigure[gnu.pdf][width=2.5cm]} {d}
     {\externalfigure[hacker.pdf][width=2.5cm]} {e}
     {\externalfigure[ctanlion.pdf][width=2.8cm]} {f}
   \stopcombination}

Well, that's all for figures.  Happy playing with them!  BTW., you may
have noticed that \type{[bottom]} is used in the block of code above.
Pay attention to its behavior.

\section{Tables}

The construction of a tables in \CONTEXT\ might be quite different
with what in your mind, especially if you have used \LaTeX\ before.
Let's go through an example first.

\startTEX
\starttable[|l|c|r|]
\NC {\bf Artist} \VL {\bf Album} \NC {\bf Title} \LR
\HL
\NC Pink Floyd \NC Wish You Were Here  \VL Have A Cigar \AR
\NC Muse \NC Hullabaloo \VL Forced In \AR
\NC Queen \NC A Night at The Opera \VL Death on Two Legs \LR
\stoptable
\stopTEX

\starttable[|l|c|r|]
\NC {\bf Artist} \VL {\bf Album} \NC {\bf Title} \LR
\HL
\NC Pink Floyd \NC Wish You Were Here  \VL Have A Cigar \AR
\NC Muse \NC Hullabaloo \VL Forced In \AR
\NC Queen \NC A Night at The Opera \VL Death on Two Legs \LR
\stoptable

As one can see, the formatting string which enclosed in the pair of
square brace tells \CONTEXT\ only the align of the tabel, though we
use \type{|} to seperate the three charactors.  In \CONTEXT, we apply
a more flexible scheme to typeset vertical lines in tables in
comparison to that in \LaTeX\ that we use \type{\VL} to separate two
cells so that \CONTEXT\ will add a vertical line between them.

If you do not prefer a vertical line there, use \type{\NC} instead of
\type{\VL}.  All these “two-cap-lettered” macros for typesetting
tables are listed in table.  To end a row, we usually use a \type{\AR}
macro or, instead, \type{\LR} if it is either the last row or followed
by a \type{\HL} that draw a horizontal line across the table.

\CONTEXT\ provides a whole bunch of macros and options for column and
cell definition.  See a full list of them at the \from[wiki],
\from[wiki_table].

Like a figure, a table can be a floating object whose placement is
calculated by \CONTEXT.

\placetable[right][tab:cell-def]{Cell Definition Macros}{
  \starttable[|c|l|]
    \NC {\bf Macro} \VL {\bf Meaning} \LR
    \HL
    \NC \type{\NC} \VL next column \AR
    \NC \type{\HL} \VL horizontal line \AR
    \NC \type{\VL} \VL vertical line \AR
    \NC \type{\NR} \VL next row \LR
    \HL
    \NC \type{\SR} \VL single row \AR
    \NC \type{\FR} \VL first row \AR
    \NC \type{\MR} \VL middle row \AR
    \NC \type{\LR} \VL last row \LR
    \HL
    \NC \type{\AR} \VL automatic row \SR
  \stoptable}
\startTEX
\placetable[right][tab:cell-def]{Cell Definition Macros}{
  \starttable[|c|l|]
    \NC {\bf Macro} \VL {\bf Meaning} \LR
    \HL
    \NC \type{\NC} \VL next column \AR
    \NC \type{\HL} \VL horizontal line \AR
    \NC \type{\VL} \VL vertical line \AR
    \NC \type{\NR} \VL next row \LR
    ...
  \stoptable}
\stopTEX

\section{Descriptions}

What you will write if you want to when you want to define something
or describe stuff your readers might be interested in?  Well,
an “aaa: bbb” pair seems stupid enough to demonstrate that the
writer is without sufficient training on typesetting.  \CONTEXT\
enables us to make it more fancy.  To start with, we first use
\type{\definedescription} macro:
\startTEX
\definedescription[Fractal][]
\stopTEX
which defines a stuff we would like to describe in default style.
Then we can describe it like this
\startTEX
\definition{Fractal}
A fractal is an object or quantity that displays self-similarity, in a somewhat technical sense, on all scales.\par
\stopTEX
\definition{Fractal}
A fractal is an object or quantity that displays self-similarity, in a
somewhat technical sense, on all scales. \par

Note that the trailing \type{\par} or an extra newline is a must to
end the description.  If the description contains multiple paragraphs,
we can also go for an environment:

\startTEX
\startdefinition{Vimperator}
Vimperator is a free browser add-on for Firefox, which makes it look and behave like the Vim text editor. It has similar key bindings, and you could call it a modal web browser, as key bindings differ according to which mode you are in.

Vimperator was written by Martin Stubenschrott. If you appreciate my work on Vimperator and want to encourage me working on it more, you can either send me greetings, patches or make a donation.
\stopdefinition
\stopTEX

\startdefinition{Vimperator}
  Vimperator is a free browser add-on for Firefox, which makes it look
  and behave like the Vim text editor. It has similar key bindings,
  and you could call it a modal web browser, as key bindings differ
  according to which mode you are in.

  Vimperator was written by Martin Stubenschrott. If you appreciate my
  work on Vimperator and want to encourage me working on it more, you
  can either send me greetings, patches or make a donation.
\stopdefinition

We have already gone through lists in section \in[sec:lists], which is
for normal vanilla itemizaion or enumerization.  In this section, a
more powerful technique is introduced to typeset item-structured
content.  The \type{itemize} environment provides us a good start for
this kind of stuff, but what if we want a list with prefix other than
table \in[tab:listprefix] has exhibited?  \CONTEXT\ surely has macros
for this, one of which is \type{\definedescription} macro.  To use
this command, we may first define our own description style like this

\startTEX
\definedescription
  [myitemization]
  [location=serried,headstyle=bold,width=broad]
\stopTEX

Then we have the macro \type{\myitemization} to typeset the
description as we want:

\startTEX
\myitemization{icon}
What for some languages looked like a handicap has now become a feature.  Thousands of words and concepts are already layed down in characters.  These characters therefore can be considered icons. \par
\stopTEX

\startnarrower
\definedescription
  [myitemization]
  [location=serried,headstyle=bold,width=broad]
\myitemization{icon}
What for some languages looked like a handicap has now become a feature.
Thousands of words and concepts are already layed down in characters.
These characters therefore can be considered icons. \par
\stopnarrower

Even better, numerical index can be added if we use
\type{\defineenumeration} to define our listing style:

\startTEX
\defineenumeration
  [remark]
  [location=top, text=Remark, between=\blank,
   before=\blank, after=\blank]
\stopTEX

\startTEX
\remark The Mod Showroom is only for posting previews/downloads of completed or near finished mods. \par
\subremark All help topics should be posted in the Editing Discussion forums. Help topics and mod requests posted here will be locked or binned. Thank you. \par
\remark Spaceeinstein's All In One Mod v2, A lot more mods rolled into one. \par
\stopTEX

\defineenumeration
  [remark]
  [location=top, text=Remark, between=\blank,
   before=\blank, after=\blank]

\remark The Mod Showroom is only for posting previews/downloads of
completed or near finished mods. \par
\subremark All help topics should be posted in the Editing Discussion
forums. Help topics and mod requests posted here will be locked or
binned. Thank you. \par
\remark Spaceeinstein's All In One Mod v2, A lot more mods rolled into
one. \par


\section{Build a Dictionary}

\ldots\ or not.  Ok, this section is not about how to build a
dictionary, though the underlying principle is no different --
indexing.  This is not a feature that every book needs.  But when
provided in a proper book, it is usually coming handy.  In \CONTEXT,
indexing is really made easy.  All you have to do is inserting a
\type{\index} at wherever the phrase you want to index is, and
placeing a \type{\placeindex} where you want the glossary be.  For
example

\startTEX
For over a thousand generations the Jedi Knights\index{Jedi} were the guardians of peace and justice in the Old Republic\index[old republic]{the old republic}. Before the dark times, before the Empire\index[empire]{the empire}.
\stopTEX

This renders just like a normal paragraph:

\startnarrower
For over a thousand generations the Jedi Knights\index{Jedi} were the
guardians of peace and justice in the Old Republic\index[old
  republic]{the old republic}. Before the dark times, before the
Empire\index[empire]{the empire}.
\stopnarrower

Note that the words in the square braces (if any) are those that are
taken into account when indexing.  Now let's place the glossary

\startTEX
\placeindex
\stopTEX

\placeindex

Like many other \CONTEXT\ commands, users can define their own series
of indexing, which pluses the default \type{\index} series are called
register.  A style of register can be defined using
\type{\defineregister}

\startTEX
\defineregister[refer][refers]
\stopTEX

in which the second argument is the plural form of the name.  Then we
can use \type{\refer} and \type{\placerefer} just as \type{\index}.

\section{Page Layout}

Here comes the final section, in which we will deal with some layout
issue.  The most import command, of course, is \type{\setuplayout}.  I
usually put one in the third line of any \CONTEXT\ document while the
first line is probably a \type{\usemodule} and the second is
\type{\setuppapersize}.

Generally, there are five areas that we put text into: main text,
header, footer, left margin and right margin.  Note that adjusting
header and footer will influence the position and height of main text,
while margins will not.  The dimensions of a page from top to bottom
are \type{topspace}, \type{header}, \type{footer}, \type{bottomspace}.
Practically, I usually set \type{width} and \type{height} to
\type{fit} which means they are calculated automatically, and adjust
the four dimensions for vertical placment, and \type{backspace} for
horizontal position of main text, which gives the distance between the
left edge of the page and that of the main text area.  Margins seem to
have nothing to do with it.  They are adjusted independently using
\type{*margin} and \type{*margindistance} in which \type{*} could be
either \type{left} or \type{right}.  The layout I use for this
document is shown below.

\startTEX
\setuppapersize[S6][S6]
\setuplayout [width=fit, height=fit,
              rightmargin=1.5cm,
              leftmargin=1.5cm,
              leftmargindistance=0pt,
              rightmargindistance=0pt,
              topspace=1.8cm,
              header=0pt,
              footer=1.2cm,
              bottomspace=0.5cm,
              backspace=1.5cm,
              location=singlesided]
\stopTEX

As you can see, I leave no space for header text and $1.2\,{\rm cm}$
for footer (pape number).  While you are tuning layout,
\type{\showlayout} and \type{\showframe} can be a {\em big} help.

Now let's start setting up header and footer.  They are generally the
same except the position, so I will focus on footer, which can be
configured with \type{\setupfootertexts} command.  This macro may take
one, two, three, four or five arguments.

\startitemize[n]
\item If only one argument is issued, it place the text in the middle
  of the page (as in this document), for example

\startTEX
\setupfootertexts{pagenumber}
\stopTEX

This argument can also be \type{date}, a section name (like
\type{chapter} or \type{subsection}), name of a mark, or some macros
that will be substituted with texts (like \type{{\pagenumber of
\totalnumberofpages}}).

\item If two arguments are given, the first one will be placed on the left
and the second one right.

\item If three arguments are given, the first one will represent the
  location of the footer text, and can be one of \type{text},
  \type{margin} and \type{edge} (outside the margins).

\item Four arguments can only be given when writing a double-sided
  document.  They will be placed at even page left, even page right,
  odd page left and odd page right, respectively.

\item If five arguments are given, the first one will represent the
  location of the texts.  Despite this, it is identical to the
  condition in which four arguments are issued.

\stopitemize

\section{Conclusion}

\CONTEXT\ rocks!

\stoptext

%%% Local Variables: 
%%% mode: context
%%% TeX-master: t
%%% TeX-PDF-mode: t
%%% End: 
